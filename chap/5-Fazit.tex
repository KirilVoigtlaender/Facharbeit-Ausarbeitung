Der Röhrenfernseher dient zu einer guten Darstellung des Verhaltens von geladenen Teilchen in elektrischen und magnetischen Feldern.
Dabei lässt sich bei der Elektronenkanone der Einfluss von elektrischen Feldern beobachten.  
Durch die Ablenkspulenpaare lässt sich der Einfluss von magnetischen Feldern auf bewegte geladene Teilchen betrachten.
Beide Einrichtungen zusammen erlauben dadurch die Darstellung der wichtigen Effekte in diesem Teilgebiet der Physik.
Durch die in dieser Facharbeit erstellte Simulation lassen sich diese Effekte "`erleben"'.

In der Praxis gehört zum Fernseher mehr als nur die Bildröhre.
Zum Beispiel muss eine Zerlegung, Ausstrahlung und Empfang des Bildsignals durch Radiowellen oder andere Übertragungswege erfolgen.
Die dafür verwendete Physik war kein Teil dieser Arbeit.
Als Pionier für diese Aspekte des Fernsehens gilt der Physiker und Techniker Manfred von Ardenne (welcher nach seiner Mitarbeit am sowjetischen Atombombenprojekt jahrelang ein physikalisches Forschungsinstitut in Dresden am Geburtsort des Autors dieser Facharbeit leitete).

Abschließend lässt sich anmerken, dass fast alle heutigen Fernseher nicht mehr mit der Bildröhrentechnik arbeiten.
Stattdessen kommen neuere physikalische Erkenntnisse zum Einsatz, wie zum Beispiel Einsteins Photoeffekt in LED-Fernsehern.