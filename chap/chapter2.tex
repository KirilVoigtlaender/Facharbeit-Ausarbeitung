\section{B-Feld:}
\subsection{Entstehung und Beschreibung}
Als Grundlage, dass ein E-Feld entsteht muss eine Ladung genommen werden.
Zuerst muss der Begriff von Ladung geklärt werden:
Ladung ist eine Eigenschaft von Materie.
Die Einheit der Ladung ist 1$C$ (Coulomb).
Alle bisherigen Experimente legen nahe, dass es nur 2 Arten von elektrischer Ladung gibt; Plus und Minus.
Jede Ladung wird von einem elektrischen Feld umgeben.
Die Richtung der Feldlinien ist immer die Richtung der Kraftwirkung auf einen positiv geladenen Probekörper.
Eine Ladung wirkt eine Kraft auf eine andere Ladung.
Gleich geladene Ladungen stoßen sich ab, während unterschiedlich geladene Ladungen sich anziehen.
Diese Kraft wird als elektrische Kraft bezeichnet oder genauer als Coulomb-Kraft.
Die Herleitung der Coulomb-Kraft sieht wie folgt aus:
Man nehme sich zwei Punktladungen, welche mit $q_1$ und mit $q_2$ gekennzeichnet werden. 
Des weiteren ist ein Abstand $r$ vorhanden.
Es gibt einmal die Kraft von $q_1$ auf $q_2$ ($\vv{F_{12}}$) und einmal die Kraft von $q_2$ auf $q_1$ ($\vv{F_{21}}$).
Diese beiden Kräfte sind nach Newton gegengesetzt und zudem betragsgleich.
Durch betrachten der Animation\cite{Animation} 
lässt sich schließen, dass wenn $r$ gleich bleibt, aber die die Ladungen $q_1$ und $q_2$ erhöht werden, die Coulomb-Kraft ebenfalls zunimmt.
Daraus lässt sich schließen, dass die elektrische Kraft proportional zu den Ladungen $q_1$ und $q_2$ ist. 
Wenn die Ladungen $q_1$ und $q_2$ gleich bleiben und $r$ erhöht wird, dann verringert sich die elektrische Kraft.
Daraus lässt sich wiederum schließen, dass die elektrische Kraft antiproportional zum Abstand $r$ ist.
Daraus ergibt sich die Formel: $F_{el} = \frac{q_1 \cdot q_2}{r^2}$.
Diese Kraftauswirkung geschieht in einem Feld, welches als elektrisches Feld bekannt ist.
Die Formel für das E-Feld ist : $E = \frac{F}{q}$.
Die Definition des E-Feldes lautet: $[E] = \frac{N}{C} = \frac{V}{m}$.

%Beschreibung
Das elektrische Feld ist eine Eigenschaft des Raums, die Kraftwirkung zwischen zwei Körpern ohne materielle Verbindung beschreibt und durch virtuelle Feldlinien dargestellt werden kann. 
Die Egenschaften des elektrischen Feldes werden mithilfe von Probeladung und Feldlinien beschrieben.
Die elektrischen Feldlinien verlaufen stehts von Plus nach Minus, dies ist wie oben bereits genannt wegen der Kraftwirkung auf einen positiven Probekörper.
Die Feldlinien kreuzen und berühren sich nicht.
Wenn die Feldlinien eng beieinander liegen (hohe Dichte), dann ist das dort auch existierende elektrische Feld stark.
Wenn die Dichte der Feldlinien allerdings niedrig ist, dann ist das elektrische Feld auch schwach.
Es gibt zwei unterschiedliche Arten von elektrischen Feldern. 
Es gibt zum einen das homogene elektrische Feld und zum anderne das inhomeogene elektrische Feld.
Bei dem homogenen elektrischen Feld stehen die Feldlinien parallel zu einander und das elektrische Feld ist an allen Stellen gleich stark.
Bei dem inhomogenen elektrischen Feld kann diese Annahme nicht getätigt werden.

\subsection{Anwendung}
 
\section{E-Feld:}
\subsection{Entstehung und Beschreibung}
\subsection{Anwendung}
\section{Wechselwirkung der Felder}%Maxwell-Gleichung