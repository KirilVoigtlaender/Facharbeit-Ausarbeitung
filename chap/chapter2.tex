Text von Kapitel 2.
\section{B-Feld:}
\subsection{Entstehung und Beschreibung}
Als Grundlage, dass ein E-Feld entsteht muss eine Ladung genommen werden. Ladung ist eine Eigenschaft von Materie.
Die Einheit der Ladung ist 1$C$ (Coulomb).
Alle bisherigen Experimente legen nahe, dass es nur 2 Arten von elektrischer Ladung gibt; Plus und Minus.
Jede Ladung wird von einem elektrischen Feld umgeben.
Die Richtung der Feldlinien ist immer die Richtung der Kraftwirkung auf einen positiv geladenen Probekörper.
%Beschreibung
Das elektrische Feld ist eine Eigenschaft des Raums, die Kraftwirkung zwischen zwei Körpern ohne materielle Verbindung beschreibt und durch virtuelle Feldlinien dargestellt werden kann. 
Die Egenschaften des elektrischen Feldes werden mithilfe von Probeladung und Feldlinien beschrieben.
Die elektrischen Feldlinien verlaufen stehts von Plus nach Minus, dies ist wie oben bereits genannt wegen der Kraftwirkung auf einen positiven Probekörper.
Die Feldlinien kreuzen und berühren sich nicht.
Wenn die Feldlinien eng beieinander liegen (hohe Dichte), dann ist das dort auch existierende elektrische Feld stark.
Wenn die Dichte der Feldlinien allerdings niedrig ist, dann ist das elektrische Feld auch schwach.
Es gibt zwei unterschiedliche Arten von elektrischen Feldern. 
Es gibt zum einen das homogene elektrische Feld und zum anderne das inhomeogene elektrische Feld.
Bei dem homogenen elektrischen Feld stehen die Feldlinien parallel zu einander und das elektrische Feld ist an allen Stellen gleich stark.

\subsection{Anwendung}

\section{E-Feld:}
\subsection{Entstehung und Beschreibung}
\subsection{Anwendung}
\section{Wechselwirkung der Felder}%Maxwell-Gleichung