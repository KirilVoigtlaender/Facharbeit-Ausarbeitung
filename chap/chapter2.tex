\section{E-Feld:}
\subsection{Entstehung und Beschreibung}
Jede Ladung wird von einem elektrischen Feld umgeben.
Zuerst muss der Begriff von Ladung geklärt werden:
Ladung ist eine Eigenschaft von Materie.
Die Einheit der Ladung ist 1$C$ (Coulomb).
Alle bisherigen Experimente legen nahe, dass es nur 2 Arten von elektrischer Ladung gibt; Positiv und Negativ.
Die Richtung der Feldlinien ist immer die Richtung der Kraftwirkung auf einen positiv geladenen Probekörper.
Eine Ladung bewirkt eine Kraft auf eine andere Ladung.
Gleich geladene Ladungen stoßen sich ab, während entgegengesetzte geladene Ladungen sich anziehen.
Diese Kraft wird als elektrische Kraft bezeichnet oder genauer als Coulomb-Kraft.
Die Herleitung der Coulomb-Kraft sieht wie folgt aus:
Man nehme sich zwei Punktladungen, welche mit $q_1$ und mit $q_2$ gekennzeichnet werden. 
Des Weiteren ist ein Abstand $r$ zwischen den Ladungen vorhanden.
Es gibt einmal die Kraft von $q_1$ auf $q_2$ ($\vv{F_{12}}$) und einmal die Kraft von $q_2$ auf $q_1$ ($\vv{F_{21}}$).
Durch betrachten der Formel: $F_{el} = \frac{q_1 \cdot q_2}{r^2}$ 
lässt sich schließen, dass wenn $r$ gleich bleibt, aber die die Ladungen $q_1$ und $q_2$ erhöht werden, die Coulomb-Kraft ebenfalls zunimmt.
Daraus lässt sich schließen, dass die elektrische Kraft proportional zu den Ladungen $q_1$ und $q_2$ ist. 
Wenn die Ladungen $q_1$ und $q_2$ gleich bleiben und $r$ erhöht wird, dann verringert sich die elektrische Kraft.
Daraus lässt sich wiederum schließen, dass die elektrische Kraft anti proportional zum Abstand $r$ ist.
Diese zuvor beschriebene Kraftwirkung, lässt sich mittels Probeladungen im ganzen Raum um die Ladung messen und als Vektor an jedem Punkt visualisieren.
Diese Visualisierung des Raumes nennt man elektrisches Feld.
Die jeweilige Stärke ergibt sich aus der Größe der Probeladung "`q"' und der Größe der Kraft "`F"' auf die Ladung.
Die Dimension für das E-Feld ist : $E = \frac{F}{q}$.
Die Definition des E-Feldes lautet: $[E] = \frac{N}{C} = \frac{V}{m}$.
Veranschaulichen lässt sich dies durch die Animation \cite{Animation}.

%Beschreibung
Wenn ein elektrisches Feld vorhanden ist, lässt sich dieses als Eigenschaft des Raums auffassen.
Die Kraftwirkung des elektrischen Feldes wird mit Probeladung ermittelt und durch Feldlinien visuell dargestellt.
Die Feldlinien kreuzen und berühren sich nicht.
Wenn die Feldlinien eng beieinander liegen (hohe Dichte), dann ist das dort existierende elektrische Feld stark.
Wenn die Dichte der Feldlinien allerdings niedrig ist, dann ist das elektrische Feld schwach.
Es gibt zwei unterschiedliche Arten von elektrischen Feldern. 
Es gibt zum einen das homogene elektrische Feld und zum anderen das inhomogene elektrische Feld.
Bei dem homogenen elektrischen Feld stehen die Feldlinien parallel zu einander und das elektrische Feld ist an allen Stellen gleich stark.
Bei dem inhomogenen elektrischen Feld kann diese Annahme nicht getätigt werden.
% hier eventuell noch auf die verschiedene Ladungen eingehen (Punktladung, etc.)
\subsection{Anwendung}
Als Anwendungsbeispiel lässt sich der Plattenkondensator nehmen:
Hier lässt sich die Annahme treffen, dass das elektrische Feld zwischen den Platten homogen ist.
Des Weiteren ist das elektrische Feld als $E = \frac{U}{d}$ definiert.
Somit nimmt das elektrische bei einer größeren Spannung "`U"' zu und bei größerem Abstand der Platten "`d"' ab.
Beim Kondensator kann noch eine weitere Größe eingeführt werden, die Kapazität "`C"'.
Die Kapazität ist dabei so definiert, dass sie die maximale Menge an Ladung, die der Kondensator auf den Platten speichern kann, angibt.
Die Herleitung ist wie folgt:
$\mbox{Flächenladungsdichte: } \sigma \mbox{:=} \frac{Q}{A}$.
Dies führt zum elektrischen Feld mit $\sigma = \epsilon_0 \cdot E$ als Feldgleichung.
Dabei beträgt $\epsilon_0 = 8.854 \cdot 10^{-12} \frac{As}{Vm}$ und ist unter dem Namen "`Dielektrizitätskonstante"' bekannt.
Im allgemeinen ist das Feld auch noch vom jeweiligen Material abhängig, wodurch die Gleichung einen weiteren Term "`$\epsilon_r$"'enthält "`$\sigma = \epsilon_0 \cdot \epsilon_r \cdot E$ "'.
Diese zusätzliche Konstante berücksichtigt dabei die spezifische Materialeigenschaften und nennt sich "` relative Permeabilität"'.
Im Vakuum ist diese jedoch = 1.
Daraus folgt, dass $ Q = \epsilon_0 \cdot \epsilon_r \cdot \frac{A}{d} \cdot U$ ist.
Die Kapazität ist als $\mbox{C:= } \frac{Q}{U} = \epsilon_0 \cdot \epsilon_r \cdot \frac{A}{d}$ definiert.
Des Weiteren kann die Energie im Plattenkondensator bestimmt werden.
Dafür wird angenommen, dass der Kondensator zuerst neutral ist und eine äußere Spannung $U_0$ anliegt.
Im ersten Schritt fließt eine kleine Portion Ladung $\Delta Q$ mit dem Energieaufwand $\Delta W_1$ auf die andere Platte. Dabei entsteht die Spannung $U_1$ zwischen den Platten $\rightarrow$ $\Delta W_1 = U_1 \cdot \Delta Q$.
Es wird mehr Energie benötigt, um  weitere gleichgroße Ladungsportionen $\Delta Q$ auf die Platte zu bringen $\Delta W_2 = U_2 \cdot \Delta Q$.
Die letzte Ladungsportion $\Delta Q$ wird mit der Energie $\Delta W_E$ auf die Platte transportiert und dabei liegt zwischen den Platten die Spannung $U_0$ an. 
$\Delta W_E = U_0 \cdot \Delta Q$.
Daraus folgt die Gesamtenergie: W = $\Delta W_1 + \Delta W_2 +...+ \Delta W_E$
$$\Rightarrow W \mbox{ = } \frac{1}{2}\cdot Q_E \cdot U_o \mbox{ wegen $Q_E = C \cdot U_0$}$$
$$W= \frac{1}{2} \cdot C \cdot U_o^2$$
\section{B-Feld:}
\subsection{Entstehung und Beschreibung}
Als Grundlage, dass ein B-Feld entsteht muss der Magnetismus als Grundlage genommen werden.
Dafür muss der Begriff des Magnetismus zu erst geklärt werden.
%Magnetismus erklären
Die wirkende Kraft wird als magnetische Kraft bezeichnet oder als Lorentzkraft, wenn es sich um einen einzelnen Ladungsträger handelt.
Die Herleitung der Lorentzkraft sieht wie folgt aus:
Man nehme sich einen Ladungsträger (negativ) und einen langen Leiter.
Die Formel für die magnetische Kraft lautet: $F_{mag} =  B \cdot I_L \cdot \Delta l$.
Der Strom $I_L$ kann durch $\frac{\Delta Q}{\Delta t}$ ersetzt werden.
Dadurch kommt die Formel $F_{mag} = B \cdot \frac{\Delta Q}{\Delta t} \cdot l$ raus. 
Als nächstes lässt sich $\Delta Q$ durch $N \cdot e$ ersetzten.
Dies führt zu der Formel $F_{mag} = B \cdot N \cdot e \frac{\Delta l}{\Delta t}$.
Dabei lässt sich der Quotient als $v$ vereinfachen, was die Geschwindigkeit der Ladungsträger angibt.
Dadurch beträgt die Formel für die magnetische Kraft nun: $F_{mag} = B \cdot N \cdot e \cdot v$.
Allerdings soll die Lorentzkraft hergeleitet werden, welches die Kraft auf einen einzelnen Ladungsträger darstellt, weshalb die magnetische Kraft mit $N$ dividiert werden muss.
Dies führt dazu, dass die Formel für die Lorentzkraft $F_L = B \cdot e \cdot v$ ist.
Diese Formel lässt sich für jede beliebige Ladung benutzen, weshalb $e$ durch $q$ ersetzt werden kann.
Daraus ergibt sich die Formel: $F_L = q \cdot B \cdot v$.
Der Vektor der Lorentzkraft senkrecht auf der durch den Geschwindigkeitsvektor und dem Vektor der Flussdichte aufgespannten Ebene \cite{Lorentzkraft}.% Beschreibung von B-Feld
\subsection{Anwendung}
\section{Wechselwirkung der Felder}%Maxwellgleichung