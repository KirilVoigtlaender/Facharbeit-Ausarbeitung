\section{E-Feld:}
\subsection{Entstehung und Beschreibung}
Jede Ladung wird von einem elektrischen Feld umgeben.
Zuerst muss der Begriff von Ladung geklärt werden:
Ladung ist eine Eigenschaft von Materie.
Die Einheit der Ladung ist 1$C$ (Coulomb).
Alle bisherigen Experimente legen nahe, dass es nur 2 Arten von elektrischer Ladung gibt; Positiv und Negativ.
Die Richtung der Feldlinien ist immer die Richtung der Kraftwirkung auf einen positiv geladenen Probekörper.
Eine Ladung bewirkt eine Kraft auf eine andere Ladung.
Gleich geladene Ladungen stoßen sich ab, während entgegengesetzte geladene Ladungen sich anziehen.
Diese Kraft wird als elektrische Kraft bezeichnet oder genauer als Coulomb-Kraft.
Die Herleitung der Coulomb-Kraft sieht wie folgt aus:
Man nehme sich zwei Punktladungen, welche mit $q_1$ und mit $q_2$ gekennzeichnet werden. 
Des Weiteren ist ein Abstand $r$ zwischen den Ladungen vorhanden.
Es gibt einmal die Kraft von $q_1$ auf $q_2$ ($\vv{F_{12}}$) und einmal die Kraft von $q_2$ auf $q_1$ ($\vv{F_{21}}$).
Durch betrachten der Formel: $F_{el} = \frac{q_1 \cdot q_2}{r^2}$ 
lässt sich schließen, dass wenn $r$ gleich bleibt, aber die die Ladungen $q_1$ und $q_2$ erhöht werden, die Coulomb-Kraft ebenfalls zunimmt.
Daraus lässt sich schließen, dass die elektrische Kraft proportional zu den Ladungen $q_1$ und $q_2$ ist. 
Wenn die Ladungen $q_1$ und $q_2$ gleich bleiben und $r$ erhöht wird, dann verringert sich die elektrische Kraft.
Daraus lässt sich wiederum schließen, dass die elektrische Kraft anti proportional zum Abstand $r$ ist.
Diese zuvor beschriebene Kraftwirkung, lässt sich mittels Probeladungen im ganzen Raum um die Ladung messen und als Vektor an jedem Punkt visualisieren.
Diese Visualisierung des Raumes nennt man elektrisches Feld.
Die jeweilige Stärke ergibt sich aus der Größe der Probeladung "`$q$"' und der Größe der Kraft "`$F_{el}$"' auf die Ladung.
Die Formel für das E-Feld ist : $E = \frac{F_{el}}{q}$.
Die Dimension des E-Feldes lautet: $[E] = \frac{N}{C} = \frac{V}{m}$.
Veranschaulichen lässt sich dies durch die Animation \cite{Animation}.

%Beschreibung
Wenn ein elektrisches Feld vorhanden ist, lässt sich dieses als Eigenschaft des Raums auffassen.
Die Kraftwirkung des elektrischen Feldes wird mit Probeladung ermittelt und durch Feldlinien visuell dargestellt.
Die Feldlinien kreuzen und berühren sich nicht.
Wenn die Feldlinien eng beieinander liegen (hohe Dichte), dann ist das dort existierende elektrische Feld stark.
Wenn die Dichte der Feldlinien allerdings niedrig ist, dann ist das elektrische Feld schwach.
Es gibt zwei unterschiedliche Arten von elektrischen Feldern. 
Es gibt zum einen das homogene elektrische Feld und zum anderen das inhomogene elektrische Feld.
Bei dem homogenen elektrischen Feld stehen die Feldlinien parallel zu einander und das elektrische Feld ist an allen Stellen gleich stark.
Bei dem inhomogenen elektrischen Feld kann diese Annahme nicht getätigt werden.
% hier eventuell noch auf die verschiedene Ladungen eingehen (Punktladung, etc.)
\subsection{Anwendung}
\label{sec:Plattenkondensator}
Als Anwendungsbeispiel lässt sich der Plattenkondensator nehmen:
Hier lässt sich die Annahme treffen, dass das elektrische Feld zwischen den Platten homogen ist.
Allerdings muss beachtet werden, dass diese Aussage nur zwischen den Platten des Kondensators gilt und das sobald man an die Ränder des Kondensators geht, diese Aussage wegbricht.
An den Rändern wirkt nämlich ein inhomogenes elektrisches Feld.
Des Weiteren ist das elektrische Feld als $E = \frac{U}{d}$ definiert.
Somit nimmt das elektrische bei einer größeren Spannung "`U"' zu und bei größerem Abstand der Platten "`d"' ab.
Beim Kondensator kann noch eine weitere Größe eingeführt werden, die Kapazität "`C"'.
Die Kapazität ist dabei so definiert, dass sie die maximale Menge an Ladung, die der Kondensator auf den Platten speichern kann, angibt.
Die Herleitung ist wie folgt:
$\mbox{Flächenladungsdichte: } \sigma \mbox{:=} \frac{Q}{A}$.
Dies führt zum elektrischen Feld mit $\sigma = \epsilon_0 \cdot E$ als Feldgleichung.
Dabei beträgt $\epsilon_0 = 8.854 \cdot 10^{-12} \frac{As}{Vm}$ und ist unter dem Namen "`Dielektrizitätskonstante"' bekannt.
Im allgemeinen ist das Feld auch noch vom jeweiligen Material abhängig, wodurch die Gleichung einen weiteren Term "`$\epsilon_r$"'enthält "`$\sigma = \epsilon_0 \cdot \epsilon_r \cdot E$ "'.
Diese zusätzliche Konstante berücksichtigt dabei die spezifische Materialeigenschaften und nennt sich "` relative Permeabilität"'.
Im Vakuum ist diese jedoch = 1.
Daraus folgt, dass $ Q = \epsilon_0 \cdot \epsilon_r \cdot \frac{A}{d} \cdot U$ ist.
Die Kapazität ist als $\mbox{C:= } \frac{Q}{U} = \epsilon_0 \cdot \epsilon_r \cdot \frac{A}{d}$ definiert.
Des Weiteren kann die Energie im Plattenkondensator bestimmt werden.
Dafür wird angenommen, dass der Kondensator zuerst neutral ist und eine äußere Spannung $U_0$ anliegt.
Im ersten Schritt fließt eine kleine Portion Ladung $\Delta Q$ mit dem Energieaufwand $\Delta W_1$ auf die andere Platte. Dabei entsteht die Spannung $U_1$ zwischen den Platten $\rightarrow$ $\Delta W_1 = U_1 \cdot \Delta Q$.
Es wird mehr Energie benötigt, um  weitere gleichgroße Ladungsportionen $\Delta Q$ auf die Platte zu bringen $\Delta W_2 = U_2 \cdot \Delta Q$.
Die letzte Ladungsportion $\Delta Q$ wird mit der Energie $\Delta W_E$ auf die Platte transportiert und dabei liegt zwischen den Platten die Spannung $U_0$ an. 
$\Delta W_E = U_0 \cdot \Delta Q$.
Daraus folgt die Gesamtenergie: W = $\Delta W_1 + \Delta W_2 +...+ \Delta W_E$
$$\Rightarrow W \mbox{ = } \frac{1}{2}\cdot Q_E \cdot U_o \mbox{ wegen $Q_E = C \cdot U_0$}$$
$$W= \frac{1}{2} \cdot C \cdot U_o^2$$
\section{B-Feld:}
\subsection{Entstehung und Beschreibung}
Als Grundlage, dass ein B-Feld entsteht muss der Magnetismus als Grundlage genommen werden.
Dafür muss der Begriff des Magnetismus zu erst geklärt werden.
Ein Magnet wird dadurch Entmagnetisiert, dass er gestoßen oder mit einer Temperatur oberhalb der Curie-Temperatur erhitzt wird $~600°C$.
Des Weiteren führt das Durchbrechen eines Magneten in der Mitte, dazu das zwei neue Magneten entstehen, es gibt keine magnetische Monopole.
Die Feldlinien eines Magneten verlaufen vom Nordpol zum Südpol.
Ein Magnetfeld bewirkt eine Kraft auf ein geladenes Teilchen.
Die wirkende Kraft wird als magnetische Kraft bezeichnet oder als Lorentzkraft, wenn es sich um einen einzelnen Ladungsträger handelt.
Die Herleitung der Lorentzkraft sieht wie folgt aus:
Man nehme sich einen Ladungsträger (negativ) und einen langen Leiter.
Die Formel für die magnetische Kraft lautet: $F_{mag} =  B \cdot I_L \cdot \Delta l$.
Der Strom $I_L$ kann durch $\frac{\Delta Q}{\Delta t}$ ersetzt werden.
Dadurch kommt die Formel $F_{mag} = B \cdot \frac{\Delta Q}{\Delta t} \cdot l$ raus. 
Als nächstes lässt sich $\Delta Q$ durch $N \cdot e$ ersetzten.
Dies führt zu der Formel $F_{mag} = B \cdot N \cdot e \cdot \frac{\Delta l}{\Delta t}$.
Dabei lässt sich der Quotient als $v$ vereinfachen, was die Geschwindigkeit der Ladungsträger angibt.
Dadurch beträgt die Formel für die magnetische Kraft nun: $F_{mag} = B \cdot N \cdot e \cdot v$.
Allerdings soll die Lorentzkraft hergeleitet werden, welches die Kraft auf einen einzelnen Ladungsträger darstellt, weshalb die magnetische Kraft mit $N$ dividiert werden muss.
Dies führt dazu, dass die Formel für die Lorentzkraft $F_L = B \cdot e \cdot v$ ist.
Diese Formel lässt sich für jede beliebige Ladung benutzen, weshalb $e$ durch $q$ ersetzt werden kann.
Daraus ergibt sich die Formel: $F_L = q \cdot B \cdot v$.
Der Vektor der Lorentzkraft senkrecht auf der durch den Geschwindigkeitsvektor und dem Vektor der Flussdichte aufgespannten Ebene \cite{Lorentzkraft}.
Magnetfelder entstehen in der Gegenwart von Dauermagneten (bestehend aus Eisen, Kobalt, Nickel oder Legierungen) oder in der Umgebung von stromdurchflossenen Leitern. 
Die Größe der magnetischen Flussdichte "`B"' ist den magnetischen Feldlinien zugeordnet.
Sie ist durch die Kraft definiert, welche ein stromdurchflossener Leiter erfährt.
Die Visualisierung des Raums nennt man magnetisches Feld.
Die jeweilige Stärke ergibt sich aus der Größe der Kraft "`$F_L$"', der Länge des Leiters "`$l$"' und der Stärke des elektrischen Stroms "`$I$"'.
Die Formel für das B-Feldd ist : $B = \frac{F_L}{I \cdot l}$.
Die Dimension des B-Feldes lautet: $[B] = \frac{N}{Am} = \frac{Vs}{m^2} = T$.

% Beschreibung von B-Feld
Wenn ein magnetisches Feld vorhanden ist, lässt sich dieses als Eigenschaft des Raums auffassen.
Die Kraftwirkung des magnetischen Feldes wird durch geladen Teilchen (Ladungen) ermittelt und durch Feldlinien visuell dargestellt.
Die Feldlinien kreuzen und berühren sich nicht.
Des Weiteren haben Sie keinen Anfangspunkt und kein Endpunkt, sondern sind stets geschlossen.
Wenn die Dichte an Feldlinien groß ist, dann ist die Magnetfeldstärke dementsprechend ebenfalls stark.
Wenn die Dichte an Feldlinien gering ist, dann ist die Magnetfeldstärke ebenfalls gering.
Es gibt verschiedene Magnete mit dementsprechend auch verschiedene Magnetfeldern. 
Zum einen liegt der Stabmagnet vor, bei welchem das magnetische Feld analog zum elektrischen Feld eines Dipols ist.
Zum anderen liegt der Hufeisenmagnet vor, welcher gleich wie der Plattenkondensator abgebaut ist.
Bei diesem ist das magnetische Feld im inneren homogen und an den Rändern inhomogen. 
\subsection{Anwendung}
\label{sec:Fadenstrahlrohr}
Als Anwendungsbeispiel lässt sich der Fadenstrahlrohr nehmen.
Hier lässt sich die Annahme treffen, dass das magnetische Feld im inneren der Helmholtz-Spule homogen ist. Des Weitern gelangen die betrachteten Elektronen mit einer Geschwindigkeit von $v=\sqrt{\frac{2 \cdot e \cdot U_B}{m}}$. Die Elektronen wurden davor in der Elektronenkanone freigesetzt.
Dies ist in dem Kapitel \ref{sec:tolle-section} bereits dargestellt. 
Der Fadenstrahlrohr kann dafür verwendet werden um die spezifische Ladung "`$\frac{e}{m}$"' eines Elektrons oder eines Ions zu bestimmen.
Dafür kann man durch die Linke-Hand-Regel die Ablenkung eines einzelnes Elektrons im Magnetfeld bestimmen. Dabei zeigt der Daumen, die Richtung der negativen Ladungsträger an, der Zeigefinger die Richtung der Feldlinien und der Mittelfinger die senkrecht dazu wirkende Lorentzkraft.
Wenn man dies nun anwendet, kommt man zu den Schluss, dass die Elektronen auf einer Kreisbahn abgelenkt werden. 
Dadurch wirkt die Lorentzkraft als Zentripetalkraft und kann gleich gesetzt werden.
$$F_L = F_z$$
$$e \cdot v \cdot B = \frac{m \cdot v^2}{r}$$
Diese Formel lässt sich als erstes durch $v$ dividieren und als Ergebnis kommt $e \cdot B = \frac{m \cdot v}{r}$.
Als nächste wird $v$ durch den oben genannten Wert %Verweis möglich?
ersetzt: $e \cdot B = m \cdot \frac{\sqrt{\frac{2 \cdot e \cdot U_B}{m}}}{r}$.
Nun lässt sich dieses Ergebnis quadrieren und durch $e$ dividieren: $e \cdot B^2 = m \cdot \frac{2 \cdot U_B}{r^2}$.
Als letztes kann die Gleichung noch durch $m$ und durch $B^2$ dividieren und als Ergebnis kommt für die spezifische Ladung "`$\frac{e}{m}$"' raus : $\frac{e}{m} = \frac{2 \cdot U_B}{ r^2 \cdot B^2}$.
Diese Formel gibt die spezifische Ladung eines Elektrons an, allerdings kann man $e$ auch durch $q$ ersetzten und dadurch die spezifische Ladung eines jeden anderen Ions berechnen.
\section{Wechselwirkung der Felder}%Maxwellgleichung
\label{sec:Maxwell}
Als erstes wird die Wechselwirkung von statischen Feldern betrachtet.
Da bei sind die elektrischen und magnetischen Felder konstant und überlagern sich ungestört.
Dabei gilt für die Kraft die Formel:$F_{Ges} = Q (E + v \times B)$.
Hier ist sowohl die elektrische als auch die Lorentzkraft wiederzufinden.
Die Lorentzkraft ergibt sich, wenn die Stärke des elektrischen Feldes "`$E$"' gleich null ist.
Die elektrische Kraft hingegen entsteht, wenn die Geschwindigkeit "`$v$"' gleich null ist.
Den auf bewegte Teilchen wirkt die Lorentzkraft und nicht die elektrische Kraft.
Es gibt keine Beeinflussung der Felder, wenn sie stationär sind. 
Die Beeinflussung tritt erst in Kraft bei veränderlichen Feldern.
Dies lässt sich mit Hilfe der sogenannten "`Maxwell-Gleichungen"' betrachten.
Diese wären:
\begin{equation}
\label{eq:div E}
    \nabla \cdot E = \frac{1}{\epsilon_o} \cdot \rho
\end{equation}
\begin{equation}
\label{eq: div B}
    \nabla \cdot B = 0
\end{equation}
\begin{equation}
\label{eq:rot E}
    \nabla \times E = - \frac{\partial B}{\partial t}
\end{equation}
\begin{equation}
\label{eq:rot B}
    \nabla \times B = \mu_0 \cdot j + \mu_0 \cdot \epsilon_0 \cdot \frac{\partial E}{\partial t}
\end{equation}
Für die Wechselwirkung zwischen elektrischen und magnetischen Feldern ist nur die Gleichungen \ref{eq:rot E} und die Gleichung \ref{eq:rot B} notwendig zu betrachten.
Die anderen beiden Gleichungen allerdings liefern trotzdem wichtige Informationen und sind die Grundlagen für die Gleichung \ref{eq:rot E} und für \ref{eq:rot B}.
Die Maxwell-Gleichungen sind wie folgt zu lesen: 
Das "`$\cdot$"' bedeutet Divergenz und beschreibt die Quelle des Feldes.
Das "`$\times$"' bedeutet Rotation und beschreibt die Rotation des Feldes.
Dies bedeutet,dass die Gleichung \ref{eq:div E} zeigt, dass die Quelle für das E-Feld die Ladung ist.
Dies lässt sich dadurch zeigen, dass sich eine Ladung genommen wird und dort ist zu sehen, dass von ihr aus die Feldlinien ausgehen, welche das elektrische Feld darstellen.
Die Gleichung \ref{eq: div B} zeigt, dass das B-Feld keine Quelle hat, da es keine Monopole gibt und die Feldlinien stets geschlossen sind und dadurch kein Anfang bestimmt werden kann.
Dies lässt sich dadurch zeigen, dass sich ein Stabmagnet genommen wird, bei welchem die Magnetfeldlinien "`rein"' und wieder "`raus"' gehen und eine geschlossene Bahn darstellen.
Deswegen kann kein Anfang erkennt werden und dem zu folge auch keine Quelle.
Die Gleichung \ref{eq:rot E} bedeutet, dass ein sich rotierendes elektrisches Feld ein entgegengesetztes sich zeitlich änderndes magnetisches Feld erzeugt.
Die Gleichung \ref{eq:rot B} zeigt, dass ein sich rotierendes magnetisches Feld ein Spannung und ein sich zeitlich änderndes elektrisches Feld erzeugt.