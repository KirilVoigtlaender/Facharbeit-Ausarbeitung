Die vorliegende Facharbeit befasst sich mit dem Modell des Röhrenfernsehers.
Röhrenfernseher sind Geräte, bei welchen ein Bild durch entweder eine elektrische oder eine magnetische Ablenkung auf einen Bildschirm gelenkt wird.
Des Weiteren wird der Röhrenfernseher in der dazu programmierten Animation bildlich dargestellt.
In dem dargestellten Programm wird dabei die magnetische Ablenkung benutzt.
Sowohl die Animation als auch ihre physikalischen Eigenschaften werden in dem Kapitel \ref{chap:sim} erläutert.
Ein Nutzer des Programms kann eine Beschleunigungsspannung in einem bestimmten Bereich wählen, die Magnetfeldstärke zweier Ablenkspulenpaare wählen und erhält Berechnungen der Werte für verschiedene physikalische Größen (Energie, Geschwindigkeit, Kraft) und geometrische Parameter (Bahnradius, Ablenkwinkel).
Diese Werte werden außerdem für eine dynamische Darstellung des Strahlenverlaufs und entstehenden Leuchtbildes benutzt.

Im ersten Kapitel nach der Einleitung wird die Theorie des E-Feldes und des B-Feldes betrachtet.
Für das E-Feld ist wichtig, dass es eine Ladung gibt, da jede Ladung ein elektrisches Feld besitzt.
Des Weiteren wird die Formel der Coulombkraft betrachtet.
Anschließend folgt eine genauere Betrachtung eines Plattenkondensators, welcher einen Spezialfall eines elektrischen Feldes darstellt.
Dabei wird die Formel für die Kapazität "`$C$"' betrachtet.
Dies geschieht im Abschnitt \ref{sec:Plattenkondensator}.
Für das B-Feld ist der Magnetismus wichtig, welcher zu der Entstehung des B-Feldes führt.
Hier wird die Herleitung der Lorentzkraft getätigt.
Anschließend folgt eine genauere Betrachtung des Fadenstrahlrohrs, welches zur Bestimmung der spezifischen Ladung "`$\frac{e}{m}$"' benutzt wird.
Diese Betrachtung geschieht im Abschnitt \ref{sec:Fadenstrahlrohr}.
Des Weiteren wird die Wechselwirkung der beiden Felder aufeinander betrachtet.
%Dabei wird ein besondere Wert auf die Maxwell-Gleichungen gelegt und vor allem auf die Gleichungen (\ref{eq:rot E}) und auf die Gleichung (\ref{eq:rot B}).
%Dabei ist festzuhalten, dass die Wechselwirkung nur bei sich veränderten Feldern auftritt und bei statischen Feldern dieser Effekt nicht vorhanden ist.
%Des Weiteren gibt es auch noch zwei weitere Gleichungen (\ref{eq:div E}) und (\ref{eq: div B}), welche aber auf die Quelle des jeweiligen Feldes eingehen und nicht auf die Wechselwirkung.
Dies geschieht im Abschnitt \ref{sec:Maxwell}.
Das folgende Kapitel \ref{chap:fern} ist in zwei Unterkapitel aufgeteilt.
Im ersten Abschnitt \ref{sec:aufbau} wird der Aufbau der Fernsehröhre betrachtet.
Dabei ist wichtig zu beachten, dass ein Röhrenfernseher aus einer Einrichtung besteht, welche die Elektronen emittiert, einer Einrichtung, welche die Elektronen steuert, eine Einrichtung, welche die Elektronen ablenkt und ein Bildschirm, wo die Elektronen als Strahl auf einen Punkt treffen, wo das Bild erzeugt wird.
Die Ablenkung kann sowohl magnetisch als auch elektrisch sein.
Die Funktionsweise der Fernsehröhre wird im Abschnitt \ref{sec:Funktionsweise} behandelt.
In diesem Kapitel wird des Weiteren auch erklärt, welche Kräfte auf das jeweilige Elektron in der jeweiligen Einrichtung wirken und welche Resultate dies nach sich zieht.
Als nächstes wird die Umsetzung der Fernsehröhre in die Animation betrachtet.
Dafür werden kurze Auszüge des Codes der Animation betrachtet und erläutert.
Des Weiteren ist stets eine physikalische Erläuterung dabei, welche die Prozesse beschreibt die in der jeweiligen Einrichtung geschehen.
