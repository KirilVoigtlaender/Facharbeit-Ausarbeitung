\section{Aufbau}
Die Fernsehröhre benötigt eine Einrichtung, welche den Elektronenstrahl erzeugt (Elektronenkanone).
Diese Einrichtung ist wie folgt aufgebaut: Kathode, die für den Elektronenstrahl notwendige Elektronen in das Vakuum emittiert, eine Elektrode, welche die Elektronen von Kathode zum Bildschirm beschleunigt, die Einrichtung zum Steuern der Stärke des Stroms und eine Anordnung die den Strahl bündelt, damit der Strahl als feiner Punkt auf den Bildschirm trifft. 
Die Kathode besteht aus einem einseitig geschlossenem Nickelröhrchen.
Des Weiteren ist eine Oberflächenschicht aus Barium- und Strontiumoxyd auf der geschlossenen Seite der Kathodenhülse.
In der Kathodenhülse ist ein Heizwendel, welcher mit Heizstrom durchflossen ist und mit Aluminiumoxyd isoliert ist.
Die Elektronen werden mit Hilfe der positiven Spannung, welche die einzelnen Elektroden des Strahlssystem gegen die Kathode aufweisen, beschleunigt.
Als nächstes wird die Steuerelektrode für den Strahlstrom betrachtet.
Die Steuerelektrode ist auch als Wehneltzylinder bekannt.
Diese Steuerelektrode ist topfförmig aufgebaut und besitzt ein kleines kreisrundes Loch im Boden des Topfes.
Dieser Topf umgibt die Kathode, damit die emittierten Elektronen das im Topfboden befindliche Loch passieren können.
Auf die Steuerelektrode folgt die Beschleunigungselektrode.
Das Durchgangsloch der Beschleunigungselektrode ist wie von der Steuerungselektrode nicht einmal ein Milimeter groß.
Des Weiteren liegt der Abstand zwischen der Beschleunigungselektrode und der Steuerelektrode ebenfalls im Milimeterbereich.
Auf die Beschleunigungselektrode folgt eine rohrföhrmige ausgebildete Elektrode.
Hier werden die Elektronen auf ihre Endgeschwindigkeit beschleunigt.
Als nächstes Stück ist die Linsenelektrode, welche dafür sorgt, dass der Strahl als feiner Punkt auf den Bildschirm trifft.
Auf die Linsenelektrode folgt eine zweite Anode, wo der Elektronenstrahl dann austritt.
Des Weiteren wird ein Bildschirm benötigt, welcher aufleuchtet wenn dieser vom Elektronenstrahl getroffen wird.
Eine dünne Aluminiumfolie deckt die Leutschicht gegen das Röhreninnere ab.
Die Leuchrtschicht ist über die Aluminiumfolie mit der leitenden Schicht verbunden, welche die Innenfläche das Kolbentrichters bedeckt.
Dadurch ist der Stromkreis geschlossen.
Die Aluminiumfolie muss glatt sein, damit sie nach vorne reflektieren kann.
Für die Bildröhre-Frontplatte wird eine sich grau eingefärbte Glassmasse verwendet, welche ungefähr 25\% des durchgehendes Lichtes absorbiert.
Und als letztes muss es entweder eine magnetische oder eine elektrische Anordnung geben.
Die elektrische Ablenkung wird durch Ablenkplatten realisiert.
Dabei gibt es zuesrt eine vertikale Ablenkung und danach eine horizontale Ablenkung.
Diese Situation wird in der gegeben Abbildung \ref{fig:Fernsehroehre} dargestellt.
Im Kapitel \ref{sec:animation} (Animation) wird mit einer magnetischen Ablenkung gearbeitet.
Dabei werden zwei Ablenkspulenpaare benutzt, welche ebenfalls zuerst vertikal und dann horizontal angeordnet sind.

\section{Funktionsweise}

\ref{fig:Fernsehröhre} \cite{Abbildung}
\cite{Fernsehroehre}
\cite{Roehrenfernsehr}
\begin{figure}
    \centering
    \includegraphics[width=.75\textwidth]{fig/Fernsehröhre.jpg}
    \caption{Abbildung des inneren der Fernsehröhre}
    \label{fig:Fernsehroehre}
\end{figure}
