\section{Aufbau}
Die Fernsehröhre benötigt eine Einrichtung, welche den Elektronenstrahl erzeugt (Elektronenkanone).
Diese Einrichtung ist wie folgt aufgebaut: Kathode, die für den Elektronenstrahl notwendige Elektronen in das Vakuum emittiert, eine Elektrode, welche die Elektronen von Kathode zum Bildschirm beschleunigt, die Einrichtung zum Steuern der Stärke des Stroms und eine Anordnung die den Strahl bündelt, damit der Strahl als feiner Punkt auf den Bildschirm trifft. 
Die Kathode besteht aus einem einseitig geschlossenem Nickelröhrchen.
Des Weiteren ist eine Oberflächenschicht aus Barium- und Strontiumoxyd auf der geschlossenen Seite der Kathodenhülse.
In der Kathodenhülse ist ein Heizwendel, welcher mit Heizstrom durchflossen ist und mit Aluminiumoxyd isoliert ist.
Die Elektronen werden mit Hilfe der positiven Spannung, welche die einzelnen Elektroden des Strahlssystem gegen die Kathode aufweisen, beschleunigt.
Als nächstes wird die Steuerelektrode für den Strahlstrom betrachtet.
Die Steuerelektrode ist auch als Wehneltzylinder bekannt.
Diese Steuerelektrode ist topfförmig aufgebaut und besitzt ein kleines kreisrundes Loch im Boden des Topfes.
Dieser Topf umgibt die Kathode, damit die emittierten Elektronen das im Topfboden befindliche Loch passieren können.


Des Weiteren wird eine Steuerelektrode benötigt, mit welcher die Stärke des Elektronenstrahls beeinflusst wird (Wehneltzylinder).

Ein Bildschirm, welcher aufleuchtet wo er getroffen wird.

Und als letztes muss es entweder eine magnetische oder eine elektrische Anordnung geben.
In der gegebenen Abbildung \ref{fig:Fernsehroehre} wird mit einer elektrischen Ablenkung gearbeitet.
Im Kapitel \ref{sec:animation} (Animation) wird mit einer magnetischen Ablenkung gearbeitet.

\section{Funktionsweise}

\ref{fig:Fernsehröhre} \cite{Abbildung}
\cite{Fernsehroehre}
\cite{Roehrenfernsehr}
\begin{figure}
    \centering
    \includegraphics[width=.75\textwidth]{fig/Fernsehröhre.jpg}
    \caption{Abbildung des inneren der Fernsehröhre}
    \label{fig:Fernsehroehre}
\end{figure}
