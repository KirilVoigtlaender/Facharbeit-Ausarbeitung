Die vorliegende Facharbeit befasst sich mit dem Modell des Röhrenfernsehrs.
Röhrenfernsehr sind Geräte bei welchem ein Bild durch entweder einer elektrischen oder einer magnetischen Ablenkung auf einen Bildschirm gelenkt werden.
Des Weiteren wird der Röhrenfernsehr in der dazu gelegten Animation bildlich dargestellt.
Sowohl die Animation als auch ihre physikalischen Eigenschaften werden in dem Kapitel \ref{chap:sim} erläutert.
Im ersten Kapitel nach der Einleitung wird die Theorie des E-Feldes und des B-Feldes betrachtet.
Für das E-Feld wichtig ist, dass es eine Ladung gibt, da um jede Ladung ein elektrisches Feld gibt.
Des Weiteren wird die Herleitung der Coulombkraft betrachtet.
Anschließend folgt eine genauere Betrachtung eines Plattenkondensators, welcher einen Spezialfall eines elektrischen Feldes darstellt.
Dabei wird auf die Formel für die Kapazität "`$C$"' und für die gespeicherte Energie im Plattenkondensator betrachtet.
Dies geschieht im Kapitel \ref{sec:Plattenkondensator}.
Für das B-Feld ist der Magnetismus wichtig, welcher zu der Entstehung des B-Feldes führt.
Hier wird die Herleitung der Lorentzkraft getätigt.
Anschließend folgt eine genauere Betrachtung des Fadenstrahlrohrs, welcher zur der Bestimmung der spezifischen Ladung "`$\frac{e}{m}$"' benutzt wird.
Diese Betrachtung geschieht im Kapitel \ref{sec:Fadenstrahlrohr}.
Des Weiteren wird die Wechselwirkung der beiden Felder aufeinander betrachtet.
%genauer erklären was passiert. geht erst nach der stunde mit dem tutor und nachdem schreiben des Kapitels
Dies geschieht im Kapitel \ref{sec:Maxwell}.
Das folgende Kapitel \ref{chap:fern} ist in zwei unter Teile aufgeteilt.
Im ersten Unterteil \ref{sec:aufbau} wird der Aufbau der Fernsehröhre betrachtet.
Dabei ist wichtig zu beachten, dass ein Röhrenfernsehr aus einer Einrichtung besteht, welche die Elektronen emittiert, einer Einrichtung, welche die Elektronen steuert, eine Einrichtung, welche die Elektronen ablenkt und einen Bildschirm, wo die Elektronen als Strahl auf einen Punkt treffen, wo das Bild erzeugt wird.
Die Ablenkung kann sowohl magnetisch als auch elektrisch sein.
Die Funktionsweise der Fernsehröhre wird im Kapitel \ref{sec:Funktionsweise} behandelt.
In diesem Kapitel wird des weiteren auch erklärt, welche Kräfte auf das jeweilige Elektron in der jeweiligen Einrichtung wirken und welche Resultate dies mit sich zieht.
