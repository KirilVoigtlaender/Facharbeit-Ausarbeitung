Die vorliegende Facharbeit befasst sich mit dem Modell des Röhrenfernsehrs.
Röhrenfernsehr sind Geräte bei welchem ein Bild durch entweder einer elektrischen oder einer magnetischen Ablenkung auf einen Bildschirm gelenkt werden.
Des Weiteren wird der Röhrenfernsehr in der dazu gelegten Animation bildlich dargestellt.
Sowohl die Animation als auch ihre physikalischen Eigenschaften werden in dem Kapitel \ref{chap:sim} erläutert.
Im ersten Kapitel wird die Theorie des E-Feldes und des B-Feldes betrachtet.
Für das E-Feld wichtig ist, dass es eine Ladung gibt, da um jede Ladung ein elektrisches Feld gibt.
Des Weiteren wird die Herleitung der Coulombkraft betrachtet.
Anschließend folgt eine genauere Betrachtung eines Plattenkondensators, welcher einen Spezialfall eines elektrischen Feldes darstellt.
Dabei wird auf die Formel für die Kapazität "`$C$"' und für die gespeicherte Energie im Plattenkondensator betrachtet.
Dies geschieht im Kapitel \ref{sec:Plattenkondensator}.
