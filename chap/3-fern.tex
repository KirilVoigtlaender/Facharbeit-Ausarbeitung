Das Kapitel bezieht sich hauptsächlich auf die Quelle \cite{Fernsehroehre}.
\section{Aufbau}
\label{sec:aufbau}
Um die Funktionsweise einer Fernsehröhre besser verstehen zu können, sollte zuerst der Aufbau untersucht werden.
Grundlegend besteht eine Fernsehröhre aus einer Elektronenkanone mit Elektronenquelle, Ablenkplatten und einem Schirm, wobei sich all diese Teile im Inneren eines evakuierten gläsernen Kolbens befinden.
Evakuiert bedeutet hierbei, dass die Luft aus dem Kolben gesaugt wurde und es somit weniger störende Einflüsse auf den Elektronenstrahl gibt.
Der Kolben einer Fernsehröhre besteht aus Glas.
Dieser setzt sich aus drei Teilen zusammen: der Bildröhren-Frontplatte, dem Kolbentrichter und dem Kolbenhals.
Die Bildröhren-Frontplatte hat einen dickwandligen Preßteil, welcher mit einen hohen Rand ausgestattet ist.
Am Rand davon ergibt sich ein Preßnaht.
Der Kolbentrichter ist mit dem Anodenanschluss verbunden.
Der Kolbenhals schließt mit einem Preßteller ab, auf welchem das Strahlsystem aufgebaut ist.
Wie eingangs erwähnt, benötigt die Elektronenkanone eine Quelle.\footnote{Die Darstellung dafür ist \ref{fig:elS}.}
Diese Quelle ist wie folgt aufgebaut: die Kathode, die für den Elektronenstrahl notwendigen Elektronen in das Vakuum emittiert, eine Elektrode, welche die Elektronen von der Kathode zum Bildschirm beschleunigt, die Einrichtung zum Steuern der Stärke des Stroms und eine Anordnung, die den Strahl bündelt, damit der Strahl als feiner Punkt auf den Bildschirm trifft. 
Die Kathode besteht aus einem einseitig geschlossenem Nickelröhrchen.
Des Weiteren ist eine Oberflächenschicht aus Barium- und Strontiumoxyd auf der geschlossenen Seite der Kathodenhülse.
Zur Freisetzung der Elektronen, existiert in der Kathodenhülse eine Heizwendel, welche mit Heizstrom durchflossen ist und mit Aluminiumoxyd isoliert ist.
Die Elektronen werden mit Hilfe der positiven Spannung, welche die einzelnen Elektroden des Strahlssystem gegen die Kathode aufweisen, beschleunigt.
Als nächstes wird die Steuerelektrode für den Strahlstrom betrachtet.
Die Steuerelektrode ist auch als Wehneltzylinder bekannt.
Diese Steuerelektrode ist topfförmig aufgebaut und besitzt ein kleines kreisrundes Loch im Boden des Topfes.
Dieser Topf umgibt die Kathode, damit die emittierten Elektronen als Strahl gebündelt werden und somit fokussiert durch das Loch des Wehneltzylinder heraustreten können.
Auf die Steuerelektrode folgt die Beschleunigungselektrode.
Auf die Beschleunigungselektrode folgt eine rohrföhrmig ausgebildete Elektrode.
Hier werden die Elektronen auf ihre Endgeschwindigkeit beschleunigt.
Als nächstes Stück kommt die Linsenelektrode ran, welche dafür sorgt, dass der Strahl als feiner Punkt auf den Bildschirm trifft.
Auf die Linsenelektrode folgt eine zweite Anode, wo der Elektronenstrahl dann austritt.
Des Weiteren wird ein Bildschirm benötigt, welcher aufleuchtet, wenn dieser vom Elektronenstrahl getroffen wird.
Eine dünne Aluminiumfolie deckt die Leuchtschicht gegen das Röhreninnere ab.
Die Aluminiumfolie muss glatt sein, damit sie nach vorne reflektieren kann.
Für die Bildröhre-Frontplatte wird eine grau eingefärbte Glasmasse verwendet, welche ungefähr 25\% des durchgehenden Lichtes absorbiert.
Und als letztes muss es entweder eine magnetische oder eine elektrische Anordnung geben.
Die elektrische Ablenkung wird durch Ablenkplatten realisiert.
Dabei gibt es zuerst eine vertikale Ablenkung und danach eine horizontale Ablenkung.
Diese Situation wird in der gegebenen Abbildung \ref{fig:Fernsehroehre} dargestellt.
Im Kapitel \ref{sec:animation} (Animation) wird mit einer magnetischen Ablenkung gearbeitet.
Dabei werden zwei Ablenkspulenpaare benutzt, wobei ebenfalls eine vertikal und eine horizontal angeordnet ist.
Wichtige Zahlenwerte für die Fernsehröhre sind zum einem die "`Diagonale"' der Bildröhren-Frontplatte, welche somit die Größe des Bildschirms angibt.
Der andere wichtige Zahlenwert ist der Ablenkwinkel.
Damit wird der Winkel angegeben, welcher zu der Bilddiagonalen gehört.

\section{Funktionsweise}
\label{sec:Funktionsweise}
Die Kathode, welche für das Emittieren der Elektronen zuständig ist, wird erhitzt und dabei treten Elektronen aus (glühelektrischer Effekt \ref{sec:tolle-section}).
Es wird eine Oberflächenschicht aus Barium- und Strontiumoxyd benutzt, da  dabei Elektronen bereits bei einer Temperatur von $700-800 ^\circ C$ emittiert werden.
Dies bedeutet, dass die Elektronenaustrittsarbeit für solche Materialien gering ist und deswegen gern benutzt wird.
Der Strahlstrom wird mit der Spannung zwischen der Steuerelektrode und der Kathode gesteuert.
Die Steuerelektrode wird dabei so in die Schaltung eingebaut, dass sie ein gegensätzliches Potential zur negativen Kathode darstellt und die Elektronen somit anzieht (beschleunigt).
Die Beschleunigungselektrode weist eine feste positive Gegenspannung zu der Kathode von einigen 1000 Volt auf. 
Die Beschleunigungselektrode hat, wie der Name bereits sagt, die Aufgabe, die Elektronen auf eine hohe Geschwindigkeit zu bringen.
Der erste Teil der Anode bewirkt mit ihrer, gegen die Kathode positiven, hohen Annodenspannung, die Hauptbeschleunigung des Strahls.
Die Linse wirkt wegen der angelegten Spannung wie ein Engpass und führt dadurch die Elektronen zusammen, was erneut zur Bündelung des Strahls beiträgt.
Diese Bündelung wird elektrostatische Fokussierung genannt.
Beim Auftreffen des Strahls auf den Schirm wird kinetische Energie hauptsächlich in Licht und Wärme umgewandelt.
Die abgebremsten Elektronen bewegen sich nun über die Aluminiumfolie als Leitungsstrom zu der positiven Hochspannungsquelle.
Da ein schwarz-weiß Bild angestrebt wird, sollen die hellen Teile des Bildes in weißem Licht aufleuchten.
Da es keinen Stoff gibt, welcher beim Elektronenaufprall weißes Licht ausstrahlt, werden verschiedene Stoffe gemischt, damit das gewünschte Weiß entsteht.
Dies führt dazu, dass auf dem Bildschirm verteilt in einem geeignetem Verhältnis gelb und blau aufleuchtende Teilchen gemischt sind.
Bei der elektrischen Ablenkung zwischen den Platten, werden die Elektronen aufgrund der elektrischen Kraft abgelenkt (siehe Abbildung \ref{fig:Fernsehroehre}).
Sobald diese den Bereich der Platten verlassen haben, fliegen sie in einer geraden Bahn weiter in Richtung des Bildschirm.
Wenn die Ablenkung allerdings magnetisch realisiert werden soll, wird dies mittels magnetischer Ablenkspulen und der Lorentzkraft erreicht.
Dabei werden die Elektronen kreisbahnförmig abgelenkt und fliegen nach dem Verlassen des Magnetfeldes ebenfalls geradlinig auf den Bildschirm zu.
Das Bild wird beim Röhrenfernseher zeilenweise erzeugt.
Bei einem gewöhnlichen Fernseher entstehen 25 Bilder pro Sekunde.
Da bei einem Röhrenfernseher ein Bild aus 625 Zeilen besteht und damit die obere Zeile ihre Helligkeit bereits verloren hat, bis die unterste Zeile sichtbar wird, muss die Bildwechselfolge erhöht werden (z.B 50 Bilder pro Sekunde).
Das Vollbild wird in zwei Halbbilder unterteilt, welche jeweils 312,5 Zeilen haben und ineinandergeschachtelt sind.
Durch die vertikalen Ablenkspulen, wird der Strahl von links nach rechts gezogen und springt am Ende der Zeile wieder nach links, jedoch eine Zeile tiefer.
Die horizontalen Ablenkspulen ziehen den Strahl von oben nach  unten.
Am Ende der Seite beginnt dieser Vorgang erneut von oben.
Dies lässt sich in der Abbildung auf der Seite \cite{Roehrenfernsehr} betrachten.
Dort ist der Strahlengang für die ersten 13 Zeilen dargestellt.
\begin{figure}[h]
    \centering
    \includegraphics[width=.75\textwidth]{fig/Fernsehröhre.jpg}
    \caption{Abbildung des Inneren der Fernsehröhre (übernommen von \cite{Abbildung}) }
    \label{fig:Fernsehroehre}
\end{figure}
