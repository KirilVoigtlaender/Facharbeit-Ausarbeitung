\section{Konstanten}

\begin{itemize}
    \item Felddurchmesser $d$ = 30 in mm
    \item Masse des Elektrons $m$ =  $9,109 \cdot 10^{-31}$ in kg
    \item Ladung des Elektrons $e$ = $1,602 \cdot 10^{-19}$ in As
    \item Bildschirmabstand $l$ = 500 in mm
\end{itemize}

\section{Zu berechnende Parameter}

\begin{itemize}
    \item Teilchengeschwindigkeit v in $\frac{km}{s}$
    \item Ergebnisvektor $\vv{v}$
    \item Bahnradius r in mm 
    \item alpha $\alpha$
    \item Ablenkungsrichtung $\vv{a}$
\end{itemize}


\section{\textbf{Graphische Ausarbeitung}}
Wann immer man ein Objekt (Fernseheröhre oder jedes andere mögliche Objekt) auf einen Bildschirm abbilden möchte, muss man aus 3D in der "`Realität"' 2D auf dem Bildschirm machen. Dafür muss man eine Koordinatentransformation machen, welche auch als Kameratransformation bekannt ist.

\underline{$\vv{a}$ = Blickrichtung}
$$x = \sqrt{\frac{1}{6}}$$
$$y = \sqrt{\frac{2}{3}}$$
$$z = -\sqrt{\frac{1}{6}}$$
$$\vv{a}=\vektor{\sqrt{\frac{1}{6}}}{\sqrt{\frac{2}{3}}}{-\sqrt{\frac{1}{6}}} $$
$x$, $y$ sollen positiv sein, $z$ soll negativ sein und der Betrag/Länge soll 1 sein. $X$ etwas kleiner als $y$.
$$y= \sqrt{\frac{2}{3}}$$
$$x= \sqrt{\frac{1}{6}}$$
\underline{$\vv{b}$:}
$z$ = 0 ; andere Werte so wählen, dass Skalarprodukt 0 ergibt
$$x= \sqrt{\frac{2}{3}}$$
$$y=-\sqrt{\frac{1}{6}}$$
$$z=0$$
$$|\vv{b}| = \sqrt{\frac{5}{6}}\neq 1\mbox{ ; also müssen $x$ und $y$ mit dem Kehrwert von $\sqrt{\frac{5}{6}}$, also mit $\sqrt{\frac{6}{5}}$ multipliziert werden.}$$
$$x' = \sqrt{\frac{12}{15}} = \sqrt{\frac{4}{5}}$$
$$y' = - \sqrt{\frac{1}{5}}$$
$$z' = 0$$
$$\vv{b}= \vektor{\sqrt{\frac{4}{5}}}{-\sqrt{\frac{1}{5}}}{0}$$
\underline{$\vv{c}$:}
$$\vv{c} = \vv{b}\times  \vv{a}$$
$$= \vektor{\frac{4}{5}}{-\sqrt{\frac{1}{5}}}{0}
      \times \vektor{\sqrt{\frac{1}{6}}}{\sqrt{\frac{2}{3}}}{-\sqrt{\frac{1}{6}}} \mbox{ = } 
     \left(\begin{array}{c} 
     \frac{\sqrt{5}}{5}\cdot\frac{\sqrt{6}}{6}\\2\cdot\frac{\sqrt{5}}{5}\cdot\frac{\sqrt{6}}{6}\\2\cdot\frac{\sqrt{5}}{5}\cdot\frac{\sqrt{6}}{6}+\frac{\sqrt{5}}{5}\cdot\frac{\sqrt{6}}{6}\end{array}\right)
$$
Wir suchen eine Matrix $A$, sodass

     $$A\cdot \vv{a}=\vektor{0}{1}{0}\mbox{ ist ;} $$
     $$A\cdot \vv{b}=\vektor{1}{0}{0}\mbox{ ist ;}$$
     $$A\cdot \vv{c}=\vektor{0}{0}{1}\mbox{ ist ;} $$
     
$$\mbox{$B$ = }\m{2\cdot\frac{\sqrt{5}}{5}}{\frac{1}{6}\sqrt{6}}{\frac{1}{30}\sqrt{30}}{\frac{-1}{5}\sqrt{5}}{\frac{1}{3}\sqrt{6}}{\frac{1}{15}\sqrt{30}}{0}{\frac{-1}{6}\sqrt{6}}{\frac{1}{6}\sqrt{30}}$$
Um auf die Matrix $A$ zu kommen, muss man nun die Matrix $B$ mit Hilfe von GeoGebra invertieren:
$$\mbox{$A$ = } \m {2\cdot\frac{\sqrt{5}}{5}}{-\frac{\sqrt{5}}{5}}{0}{\frac{\sqrt{6}}{6}}{\frac{\sqrt{6}}{3}}{-\frac{\sqrt{6}}{6}}{\frac{\sqrt{30}}{30}}{\frac{\sqrt{30}}{15}}{\frac{\sqrt{30}}{6}}$$

%\section{Simulation}
%\subsection{a}
%\subsection{b}

\begin{figure}
    \centering
    \includegraphics[width=1.25\textwidth]{fig/Fernseherröhre.PNG}
    \caption{Darstellung der Simulation}
    \label{fig:meineSimulation}
\end{figure}