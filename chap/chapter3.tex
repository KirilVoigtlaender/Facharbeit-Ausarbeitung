Text von Kapitel 3.
\section{\textbf{Herleitung von Winkel}}

\begin{tabular}{c|c|c}
     Formel Buch & Formel Block & Anmerkungen  \\
     \hline
    $\alpha = \frac{d}{r}$ &$\tan(\frac{\alpha}{2}) = \frac{d}{2\cdot r}$& wegen Kleinwinkelnäherung bei dem Buch \\
    \hline
   $r = \frac{m\cdot v}{q\cdot B}$  & $r = \frac{m\cdot v}{q\cdot B}$& alles gleich 
     
\end{tabular}

Das ist im Magneten und hier wird die Ablenkung berechnet.

\begin{lstlisting}
bahnradius 
  = strahl.elektronenmasse 
    * strahl.teilchengeschwindigkeit 
    / (strahl.elektronenladung * magnetfeldstärke/1000)
    * 1000;
alpha = 2 * Math.atan(felddurchmesser/ (2 *bahnradius ));
ablenkungsrichtung
  = strahl.quelle.richtungsvektor.
    multiplizieren(-1).
    kreuzprodukt(this.richtungsvektor);
\end{lstlisting}

Das ist im Strahl und hier wird die Geschwindigkeit berechnet mit welcher dieser aus der Elektronenkanone kommt. 

\begin{lstlisting}
teilchengeschwindigkeit 
  = Math.sqrt(2 
  * elektronenladung 
  * quelle.spannung 
  * 1000 
  / elektronenmasse);
\end{lstlisting}

Hier wird der Punkt auf dem Bildschirm berechnet.

\begin{lstlisting}
Vektor ergebnisvektor 
  = new Vektor(bildschirmabstand,0,0);
    for(Magnet m : getWorld().getObjects(Magnet.class) )
        {
            ergebnisvektor 
            = ergebnisvektor.addieren( m.ablenkungsrichtung.
            multiplizieren(bildschirmabstand).
            multiplizieren(Math.tan(m.alpha)));
        }
    return ergebnisvektor;
\end{lstlisting}
 
