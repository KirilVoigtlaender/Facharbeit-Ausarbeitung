\section{\textbf{Graphische Ausarbeitung}}

Wenn immer man ein Objekt(Fernsehröhre oder jedes andere mögliche Objekt) auf einen Bildschirm abbilden möchte, muss man aus 3D in der "Realität" 2D auf dem Bildschirm machen. Dafür muss man eine Koordinatentransformation machen, welche auch als Kameratransformation bekannt ist.

\underline{Würfel:}
$$(x -100 \mbox{bis}  100)$$
$$(y-100\mbox{bis}100)$$
$$(z-100\mbox{bis}100$$

\underline{N Nullpunkt auf dem Schirm}
$$x = -60$$
$$y = -50$$
$$z = 30$$
\underline{a = Blickrichtung}
$$x = \sqrt{1/6}$$
$$y = \sqrt{2/3}$$
$$z = -\sqrt{1/6}$$
x, y sollen positiv sein, z soll negativ sein und der Betrag/Länge soll eins sein. X etwas kleiner als y.
$$y= \sqrt{2/3}$$
$$x= \sqrt{1/6}$$
\underline{b:}
z = 0 ; andere Werte so wählen, dass Skalarprodukt 0 ergibt
$$x= \sqrt{2/3}$$
$$y=-\sqrt{1/6}$$
$$z=0$$
$$|\vv{b}| = \sqrt{5/6} =/1 ; also müssen x und y mit dem Kehrwert von\sqrt{5/6} also mit \sqrt{6/5} multipliziert werden.$$